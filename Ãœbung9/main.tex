%%%%%%%%%%%%%%%%%%%%%%%%%%%%%%%%%%%%%%%%%
% Wenneker Assignment
% LaTeX Template
% Version 2.0 (12/1/2019)
%
% This template originates from:
% http://www.LaTeXTemplates.com
%
% Authors:
% Vel (vel@LaTeXTemplates.com)
% Frits Wenneker
%
% License:
% CC BY-NC-SA 3.0 (http://creativecommons.org/licenses/by-nc-sa/3.0/)
% 
%%%%%%%%%%%%%%%%%%%%%%%%%%%%%%%%%%%%%%%%%

%----------------------------------------------------------------------------------------
%	PACKAGES AND OTHER DOCUMENT CONFIGURATIONS
%----------------------------------------------------------------------------------------

\documentclass[11pt]{scrartcl} % Font size

%%%%%%%%%%%%%%%%%%%%%%%%%%%%%%%%%%%%%%%%%
% Wenneker Assignment
% Structure Specification File
% Version 2.0 (12/1/2019)
%
% This template originates from:
% http://www.LaTeXTemplates.com
%
% Authors:
% Vel (vel@LaTeXTemplates.com)
% Frits Wenneker
%
% License:
% CC BY-NC-SA 3.0 (http://creativecommons.org/licenses/by-nc-sa/3.0/)
% 
%%%%%%%%%%%%%%%%%%%%%%%%%%%%%%%%%%%%%%%%%

%----------------------------------------------------------------------------------------
%	PACKAGES AND OTHER DOCUMENT CONFIGURATIONS
%----------------------------------------------------------------------------------------

\usepackage{amsmath, amsfonts, amsthm} % Math packages

\usepackage{listings} % Code listings, with syntax highlighting

\usepackage[english]{babel} % English language hyphenation

\usepackage{graphicx} % Required for inserting images
\usepackage{caption}
\graphicspath{{Figures/}{./}} % Specifies where to look for included images (trailing slash required)

\usepackage{booktabs} % Required for better horizontal rules in tables

\numberwithin{equation}{section} % Number equations within sections (i.e. 1.1, 1.2, 2.1, 2.2 instead of 1, 2, 3, 4)
\numberwithin{figure}{section} % Number figures within sections (i.e. 1.1, 1.2, 2.1, 2.2 instead of 1, 2, 3, 4)
\numberwithin{table}{section} % Number tables within sections (i.e. 1.1, 1.2, 2.1, 2.2 instead of 1, 2, 3, 4)

\setlength\parindent{0pt} % Removes all indentation from paragraphs

\usepackage{enumitem} % Required for list customisation
\setlist{noitemsep} % No spacing between list items

%----------------------------------------------------------------------------------------
%	DOCUMENT MARGINS
%----------------------------------------------------------------------------------------

\usepackage{geometry} % Required for adjusting page dimensions and margins

\geometry{
	paper=a4paper, % Paper size, change to letterpaper for US letter size
	top=2.5cm, % Top margin
	bottom=3cm, % Bottom margin
	left=3cm, % Left margin
	right=3cm, % Right margin
	headheight=0.75cm, % Header height
	footskip=1.5cm, % Space from the bottom margin to the baseline of the footer
	headsep=0.75cm, % Space from the top margin to the baseline of the header
	%showframe, % Uncomment to show how the type block is set on the page
}

%----------------------------------------------------------------------------------------
%	FONTS
%----------------------------------------------------------------------------------------

\usepackage[utf8]{inputenc} % Required for inputting international characters
\usepackage[T1]{fontenc} % Use 8-bit encoding

\usepackage{fourier} % Use the Adobe Utopia font for the document

%\usepackage[framed,numbered,autolinebreaks,useliterate]{mcode}

%----------------------------------------------------------------------------------------
%	SECTION TITLES
%----------------------------------------------------------------------------------------

\usepackage{sectsty} % Allows customising section commands

\sectionfont{\normalfont\bfseries} % \section{} styling
\subsectionfont{\normalfont\bfseries} % \subsection{} styling
\subsubsectionfont{\normalfont\itshape} % \subsubsection{} styling
\paragraphfont{\normalfont\scshape} % \paragraph{} styling

%----------------------------------------------------------------------------------------
%	HEADERS AND FOOTERS
%----------------------------------------------------------------------------------------

\usepackage{scrlayer-scrpage} % Required for customising headers and footers

\ohead*{} % Right header
\ihead*{} % Left header
\chead*{} % Centre header

\ofoot*{} % Right footer
\ifoot*{} % Left footer
\cfoot*{\pagemark} % Centre footer

% MY PACKAGES
%\usepackage[framed,numbered,autolinebreaks,useliterate]{mcode}
\usepackage{listings}
\usepackage{float}
\usepackage{amsmath}
\usepackage{tikz}
\usetikzlibrary{shapes,arrows,positioning}
\usepackage{hyperref} % Include the file specifying the document structure and custom commands

%----------------------------------------------------------------------------------------
%	TITLE SECTION
%----------------------------------------------------------------------------------------

\title{	
	\normalfont\normalsize
	\textsc{Universität Würzburg}\\ % Your university, school and/or department name(s)
	\vspace{25pt} % Whitespace
	\rule{\linewidth}{0.5pt}\\ % Thin top horizontal rule
	\vspace{20pt} % Whitespace
	{\huge Übung 9: Zeitdiskrete Systeme}\\ % The assignment title
	\vspace{12pt} % Whitespace
	\rule{\linewidth}{2pt}\\ % Thick bottom horizontal rule
	\vspace{12pt} % Whitespace
}

\author{\LARGE Alexander Björk, Janis Kaltenthaler} % Your name

\date{\normalsize\today} % Today's date (\today) or a custom date

\begin{document}

\maketitle % Print the title


\section*{Aufgabe 9-1: Berechnung eines diskreten Zustandsraummodells aus dem Kontinuierlichen (3 Punkte)}
\subsection*{a)}
Um das zeitkontinuierliche System in die diskrete Form zu überführen berechnen wir zunächst die Systemmatrix $A_d$ und den Eingabevektor $b_d$:

\begin{align*}
A_d&=e^{A\cdot T}=e^{\begin{bmatrix}-\frac{2}{3}&0\\0&-\frac{1}{3}\end{bmatrix}\cdot T}=\begin{bmatrix}e^{-2}&&0\\0&&e^{-1} \end{bmatrix}\approx\begin{bmatrix}0.135335&&0\\0&&0.367879 \end{bmatrix}\\
b_d&=\int_{0}^{T}e^{A\cdot\alpha}d\alpha\cdot b=\int_{0}^{3}\begin{bmatrix}e^{-\frac{2\alpha}{3}}&&0\\0&&e^{-\frac{\alpha}{3}} \end{bmatrix} d\alpha\cdot \begin{bmatrix}2\\3 \end{bmatrix}=\begin{bmatrix}\frac{3-3\cdot e^{-2}}{2}&&0\\0&&3-3\cdot e^{-1} \end{bmatrix}\cdot \begin{bmatrix}2\\3 \end{bmatrix}\\
b_d&=\begin{bmatrix}3\cdot(e^2-1)\cdot e^{-2}\\ 9\cdot(e-1)\cdot e^{-1}\end{bmatrix}\approx\begin{bmatrix}2.59399\\5.68909\end{bmatrix}
\end{align*}
Das zeitdiskrete Zustandsraummodell ist nun wie folgt gegeben:
\begin{align*}
	x(k+1)&=A_dx(k)+b_du(k),\hspace{10pt} x(0)=x_0\\
	y(k)&=c^Tx(k)+du(k)
\end{align*}
\subsection*{b)}
Die Eigenwerte von $A_d$ sind die Lösungen der charakteristischen Gleichung:
\begin{align*}
	\text{det}(A-\lambda I)&=0\\
	0&=(e^{-2}-\lambda)\cdot(e^{-1}-\lambda)\\
	\lambda_1&=e^{-2},\hspace{3pt}\lambda_2=e^{-1}
\end{align*}
Wegen $\lambda_1<1$ und $\lambda_2<1$ ist das zeitdiskrete System asymptotisch stabil.
\subsection*{c)}
Für die freie Bewegung des zeitdiskreten Systems gilt $u(k)=0$. die Bewegungsgleichung der freien Bewegung der Bewegungsgleichung des Ausgangs ist demnach:
\begin{align*}
	y(k)=c^TA^k_dx_0=\begin{bmatrix}1&&1\end{bmatrix}\cdot\begin{bmatrix}e^{-2}&&0\\0&&e^{-1} \end{bmatrix}^k\cdot\begin{bmatrix}1.5\\1.5\end{bmatrix}=1.5\cdot e^{-k}+1.5\cdot e^{-2\cdot k}\\
	y(1)\approx0.754822,y(2)\approx0.230476,y(5)\approx0.010175,y(10)\approx0.000068,y(1000)\approx0.0
\end{align*}
Wie schon in Aufgabe \textbf{b)} gezeigt ist das System asympthotisch stabil. Für $k\rightarrow\infty$ verschwindet die Eigenbewegung.


\section*{Aufgabe 9-2: Berechnung eines diskreten Zustandsraummodells aus dem Kontinuierlichen (1,5 Punkte)}
\subsection*{a)}
Für die Überführung in das diskrete Zustandsraummodell müssen zunächst die Matrizen 
\begin{align*}
A_d = e^{AT}
\end{align*}
und 
\begin{align*}
b_d = \int^T_0 e^{A\alpha}d\alpha b
\end{align*} 
berechnet werden. \\

Für $A_d$ muss die Matrixexponentialfunktion $\Phi(AT)$ gelöst werden. Da $AT$ eine allgemeine 2x2 Matrix ist, muss erst die zugehörige Jordan-Normalform $J$ mit Transformationsmatrix $V$ von $AT$ ermittelt werden. Diese sind
\begin{align*}
J = \begin{bmatrix}
-1 & 0 \\
0 & 0.5
\end{bmatrix}
\end{align*}
und
\begin{align*}
V = \begin{bmatrix}
-1 & 1\\
1 & 0
\end{bmatrix}.
\end{align*}
Die Matrixexponentialfunktion $\Phi(AT)$ kann nun berechnet werden durch
\begin{align*}
e^{AT}  &= V \cdot \text{diag}\left(e^{\lambda_1}, \dots, e^{\lambda_n}\right) \cdot V^{-1}.
\end{align*}
Die Eigenwerte der Diagonalmatrix befinden sich in der Jordanmatrix $J$.\\

Damit ist
\begin{align*}
A_d = \begin{bmatrix}
1.6487 & 1.2808 \\
0 & 0.3679
\end{bmatrix}.
\end{align*}

Eine Überprüfung durch Matlab mit 
\begin{align*}
\texttt{expm(A*T)}
\end{align*}
liefert die gleichen Ergebnisse.\\

Nach \cite{gajic2003linear} gilt für die Matrix $b_d$ nach Matrixintegration
\begin{align*}
b_d = e^{AT}\left( -e^{AT}A^{-1}+A^{-1} \right) B = \left( A_d - I \right) A^{-1}B
\end{align*}
und ergibt somit
\begin{align*}
b_d = \begin{bmatrix}
1.3140 \\
0.6321
\end{bmatrix}.
\end{align*}
Eine Überprüfung durch Matlab mit
\begin{align*}
\texttt{integral(@(alpha)expm(A*alpha), 0, Ts, 'ArrayValued', true) * B}
\end{align*}

liefert die gleichen Ergebnisse.\\

Auch die Matlabfunktion
\begin{align*}
\texttt{c2d(sys,T)}
\end{align*}
liefert die gleichen Ergebnisse.\\

Das zeitdiskrete Zustandsraummodell ist somit
\begin{align*}
x\left(k+1\right) &= \begin{bmatrix}
1.6487 & 1.2808 \\
0 & 0.3679
\end{bmatrix} x\left(k\right) + \begin{bmatrix}
1.3140 \\
0.6321
\end{bmatrix} u\left(k\right) \\
y\left(k\right) &= \begin{bmatrix}
1 & 1
\end{bmatrix} x\left(k\right).
\end{align*}


\subsection*{b)}
Die  Eigenwerte des diskreten Modells sind
\begin{align*}
\lambda_1 &= e^{\dfrac{1}{2}} \approx 1.6487 \\
\lambda_2 &= e^{-1} \approx 0.3679.
\end{align*}

Da beide Eigenwerte positiv und reell sind, ist das System instabil und erzeugt exponentielles Wachstum.



\section*{Aufgabe 9-3: Zeitdiskrete Systeme (1,5 Punkte)}
Die Ausgabe von \verb+uebung93.m+:
\begin{align*}
&\texttt{Das kontinuierliche System ist vollständig steuerbar!} \\
&\texttt{Das zeitdiskrete System mit Abtastrate 2s ist nicht vollständig steuerbar!} \\
&\texttt{Das zeitdiskrete System mit Abtastrate 3s ist vollständig steuerbar!}
\end{align*}

Das zeitdiskrete System mit einer Abtastrate von 2s hat eine Diagonalmatrix als Systemmatrix und bildet daher eine Steuerbarkeitsmatrix mit Rang 1 und ist deswegen nicht vollständig steuerbar.
Das zeitdiskrete System mit einer Abtastrate von 3s hat keine Diagonalmatrix als Systemmatrix. Die Steuerbarkeitsmatrix hat Rang 2. Das System ist daher vollständig steuerbar.


\section*{Aufgabe 9-4: Zeitdiskrete Systeme (2,5 Punkte)}

\newpage

\bibliography{biblio}
\bibliographystyle{apalike}


\end{document}
