%%%%%%%%%%%%%%%%%%%%%%%%%%%%%%%%%%%%%%%%%
% Wenneker Assignment
% LaTeX Template
% Version 2.0 (12/1/2019)
%
% This template originates from:
% http://www.LaTeXTemplates.com
%
% Authors:
% Vel (vel@LaTeXTemplates.com)
% Frits Wenneker
%
% License:
% CC BY-NC-SA 3.0 (http://creativecommons.org/licenses/by-nc-sa/3.0/)
% 
%%%%%%%%%%%%%%%%%%%%%%%%%%%%%%%%%%%%%%%%%

%----------------------------------------------------------------------------------------
%	PACKAGES AND OTHER DOCUMENT CONFIGURATIONS
%----------------------------------------------------------------------------------------

\documentclass[11pt]{scrartcl} % Font size

\input{structure.tex} % Include the file specifying the document structure and custom commands

%----------------------------------------------------------------------------------------
%	TITLE SECTION
%----------------------------------------------------------------------------------------

\title{	
	\normalfont\normalsize
	\textsc{Universität Würzburg}\\ % Your university, school and/or department name(s)
	\vspace{25pt} % Whitespace
	\rule{\linewidth}{0.5pt}\\ % Thin top horizontal rule
	\vspace{20pt} % Whitespace
	{\huge Übung 7}\\ % The assignment title
	{\Large Ausgangsrückführung, LQR}\\
	\vspace{12pt} % Whitespace
	\rule{\linewidth}{2pt}\\ % Thick bottom horizontal rule
	\vspace{12pt} % Whitespace
}

\author{\LARGE Alexander Björk, Janis Kaltenthaler} % Your name

\date{\normalsize\today} % Today's date (\today) or a custom date

\begin{document}

\maketitle % Print the title

\tikzstyle{block} = [draw, fill=blue!20, rectangle, 
    minimum height=3em, minimum width=6em]
\tikzstyle{sum} = [draw, fill=blue!20, circle, node distance=1cm]
\tikzstyle{input} = [coordinate]
\tikzstyle{output} = [coordinate]
\tikzstyle{pinstyle} = [pin edge={to-,thin,black}]
\newcommand{\inte}{$\displaystyle \int$}

\section*{Aufgabe 7-1. Ausgangsrückführung (7 Punkte)}
\subsection*{a)}
Damit die Pole des geschlossenen Regelkreises bei den geforderten Werten liegen, ergibt sich durch die Berechnung in Matlab die folgende Regelmatrix:
\begin{align*}
	K_x=\begin{bmatrix}1.6205\\-1.41\end{bmatrix}
\end{align*}
Der Vorfilter $V$ wird nun wie folgt bestimmt:
\begin{align*}
	V_x=-\bigl(C(A-BK_x)^{-1}B\bigr)^{-1}=0.052632
\end{align*}
Das resultierende Zustandsraummodell mit Zustandsrückführung erhält man nun wie folgt:
\begin{align*}
	\dot{x}(t)&=(A-BK_x)x(t)+BV_xw(t)=\begin{bmatrix}-0.24100&-0.17993\\-2.16205&-2.85900\end{bmatrix}x(t)+\begin{bmatrix}0.1052632\\0.0052632\end{bmatrix}w(t)\\
	y(t)&=Cx(t)=\begin{bmatrix}1&0\end{bmatrix}
\end{align*}
\subsection*{b)}
Das System mit Ausgangsrückführung ist stabil, falls alle Pole des geschlossenen Regelkreis mit Ausgangsrückführung einen negativen Realteil besitzen. Um dies zu Prüfen nutzen wir das Hurwitz-Kriterium. Die System Matrix $\bar{A}$ des geschlossenen Regelkreises berechnet sich wie folgt:
\begin{align*}
	\bar{A}=A-Bk_yC=\begin{bmatrix}3-2\cdot k_y&-3\\-0.1\cdot k_y-2&-3 \end{bmatrix}
\end{align*}
Das charakteristische Polynom des geschlossenen Regelkreises berechnet sich wie folgt:
\begin{align*}
	\text{det}(\bar{A}-\lambda I)=\underbrace{1}_{a_0}\cdot\lambda^2+\underbrace{2\cdot k_y}_{a_1}\cdot\lambda+\underbrace{5.7\cdot k_y-15}_{a_2}
\end{align*}
Daraus ergibt sich die Hurwitzmatrix:
\begin{align*}
	H=\begin{bmatrix}a_1&a_3\\a_0&a_2 \end{bmatrix}=\begin{bmatrix}2\cdot k_y&0\\1&5.7\cdot k_y-15 \end{bmatrix}
\end{align*}
Die zwei Hauptabschnittsdeterminanten $D_1$ und $D_2$ werden wie folgt berechnet:
\begin{align*}
	D_1&=a_1=2\cdot k_y\\
	D_2&=\text{det}\Biggl(\begin{bmatrix}2\cdot k_y&0\\1&5.7\cdot k_y-15 \end{bmatrix}\Biggr)=2\cdot k_y\cdot(5.7\cdot k_y-15)
\end{align*}
Das betrachtete System ist stabil falls alle Koeffizienten $a_i$ und alle Hauptabschnittsdeterminanten $D_i$ positiv sind. Daher ist das System stabil für $k_y>\frac{15}{5.7}\approx2.63$ und somit durch eine Ausgangsrückführung stabilisierbar.
\subsection*{c)}
Zuerst prüfen wir ob die Zustandsrückführung durch eine Ausgangsrückführung ersetzt werden kann:
\begin{align*}
	\text{Rang}\begin{bmatrix}C\\K\end{bmatrix}=2\neq\text{Rang }C=1
\end{align*}
Da die Ränge nicht gleich sind ist eine direkte Ersetzung nicht möglich.\\
Nun berechnen wir die Eigenvektormatrix $\bar{V}$ von $A-BK_x$:
\begin{align*}
	V=\begin{bmatrix}0.787112&0.065079\\-0.616810&0.997880\end{bmatrix}
\end{align*}
Die Wichtungsmatrix $W$ wird wie folgt definiert:
\begin{align*}
	W=\begin{bmatrix}1&0\\0&1\end{bmatrix}
\end{align*}
Nun ermitteln wir die Ausgangsrückführung $K_y$ nach Gleichung:
\begin{align*}
	k_y=KVW(CVW)^T\bigl((CVW)(CVW)^T\bigr)^{-1}=2.5712
\end{align*}
Um zu Prüfen ob die Güteforderungen erfüllt werden, d.h. ob das resultierende System stabil ist, berechnen wir die Eigenwerte der Systemmatrix $A-Bk_yC$:
\begin{align*}
	\lambda_1=0.066126,\hspace{3pt}\lambda_2=-5.208437
\end{align*}
Wegen $\lambda_1>0$ ist das System nicht stabil. Entsprechend passen wir die Wichtungsfunktion an, damit $\lambda_1$ genauer an den geforderten Eigenwert von $-0.1$ angenähert wird:
\begin{align*}
	W=\begin{bmatrix}2&0\\0&1\end{bmatrix}
\end{align*}
Nun erhalten wir eine Ausgangsrückführung von:
\begin{align*}
	k_y=2.6867
\end{align*}
mit den Eigenwerten:
\begin{align*}
	\lambda_1=-0.059103,\hspace{3pt}\lambda_2=-5.314262
\end{align*}
Das System ist nun dementsprechend stabil.\\
Der Vorfilter $V_y$ zur Sicherung der Sollwertfolge bei der Ausgangsrückführung berechnet sich wie folgt:
\begin{align*}
	V_y=-\bigl(C(A-Bk_yC)^{-1}B\bigr)^{-1}=0.055103
\end{align*}
Das resultierende Zustandsraummodell mit Ausgangsrückführung erhält man nun wie folgt:
\begin{align*}
	\dot{x}(t)&=(A-BK_yC)x(t)+BV_yw(t)=\begin{bmatrix}-2.3734&-3.0000\\-2.2687&-3.0000\end{bmatrix}x(t)+\begin{bmatrix}0.1102067\\0.0055103\end{bmatrix}w(t)\\
	y(t)&=Cx(t)=\begin{bmatrix}1&0\end{bmatrix}
\end{align*}

\end{document}
