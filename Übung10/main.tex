%%%%%%%%%%%%%%%%%%%%%%%%%%%%%%%%%%%%%%%%%
% Wenneker Assignment
% LaTeX Template
% Version 2.0 (12/1/2019)
%
% This template originates from:
% http://www.LaTeXTemplates.com
%
% Authors:
% Vel (vel@LaTeXTemplates.com)
% Frits Wenneker
%
% License:
% CC BY-NC-SA 3.0 (http://creativecommons.org/licenses/by-nc-sa/3.0/)
% 
%%%%%%%%%%%%%%%%%%%%%%%%%%%%%%%%%%%%%%%%%

%----------------------------------------------------------------------------------------
%	PACKAGES AND OTHER DOCUMENT CONFIGURATIONS
%----------------------------------------------------------------------------------------

\documentclass[11pt]{scrartcl} % Font size

\input{structure.tex} % Include the file specifying the document structure and custom commands

%----------------------------------------------------------------------------------------
%	TITLE SECTION
%----------------------------------------------------------------------------------------

\title{	
	\normalfont\normalsize
	\textsc{Universität Würzburg}\\ % Your university, school and/or department name(s)
	\vspace{25pt} % Whitespace
	\rule{\linewidth}{0.5pt}\\ % Thin top horizontal rule
	\vspace{20pt} % Whitespace
	{\huge Übung 10: Zufallsvariablen und Zufallsprozesse}\\ % The assignment title
	\vspace{12pt} % Whitespace
	\rule{\linewidth}{2pt}\\ % Thick bottom horizontal rule
	\vspace{12pt} % Whitespace
}

\author{\LARGE Alexander Björk, Janis Kaltenthaler} % Your name

\date{\normalsize\today} % Today's date (\today) or a custom date

\begin{document}

\maketitle % Print the title


\section*{Aufgabe 10‐1: Statistische Unabhängigkeit (2 Punkte)} 
Wir berechnen zuerst die Randverteilungsdichtefunktionen $f_X(x)$ und $f_Y(y)$:
\begin{align*}
	f_X(x)&=\int_{-\infty}^{\infty}f_{XY}(x,y)\hspace{3pt}dy=\begin{cases}\int_{0}^{\infty}e^{-(x+y)}\hspace{3pt}dy=e^{-x}, & x\ge\text{ und }y \ge0 \\0, & \text{sonst} \end{cases}\\
	f_Y(y)&=\int_{-\infty}^{\infty}f_{XY}(x,y)\hspace{3pt}dx=\begin{cases}\int_{0}^{\infty}e^{-(x+y)}\hspace{3pt}dx=e^{-y}, & x\ge\text{ und }y \ge0 \\0, & \text{sonst} \end{cases}
\end{align*}
Die Zufallsvariablen $X$ und $Y$ sind genau dann unabhängig, falls $f_X(x)\cdot f_Y(y)$ eine Verbundverteilunsgdichtefunktion von $X$ und $Y$ ist:
\begin{align*}
	f_X(x)\cdot f_Y(y)=\begin{cases}e^{-x}\cdot e^{-y}=e^{-(x+y)}, & x\ge\text{ und }y \ge0 \\0, & \text{sonst} \end{cases}=f_{XY}(x,y)
\end{align*}
Da $f_X(x)\cdot f_Y(y)=f_{XY}(x,y)$ gilt, sind die Zufallsvariablen $X$ und $Y$ statistisch unabhängig voneinander.
\section*{Aufgabe 10‐4: Erwartungswerte von Zufallsprozessen (1,5 Punkte)}
\subsection*{a)}
Der Erwartungswert $E[Z(t)]=\mu_z$ des Zufallsprozesses $Z(t)=a\cdot X(t)+b\cdot Y$ erhält man wie folgt:
\begin{align*}
	E[Z(t)]=a\cdot E[X(t)]+b\cdot E[Y(t)]=2\cdot E[X(t)]+3\cdot E[Y(t)]=6
\end{align*}
Die Varianzen der Zufallsprozesse erhält man wie folgt:
\begin{align*}
	\text{Var}[X(t)]&=E[X^2(t)]-(E[X(t)])^2=6\\
	\text{Var}[Y(t)]&=E[Y^2(t)]-(E[Y(t)])^2=-1
\end{align*}
Die Varianz des Zufallsprozesses $Z(t)$ erhält man wie folgt, wobei durch die Unkorreliertheit $\text{Cov}[X(t),Y(t)]=0$ gilt.
\begin{align*}
\text{Var}[Z(t)]=a^2\cdot\text{Var}[X(t)]+2ab\cdot\text{Cov}[X(t),Y(t)]+b^2\cdot\text{Var}[Y(t)]=15
\end{align*}
Das gewöhnliche Moment zweiter Ordnung erhalten wir mithilfe der Varianz und des Erwarungswertes wie folgt:
\begin{align*}
		E[Z^2(t)]=\text{Var}[Z(t)]+(E[Z(t)])^2=51
\end{align*}
\subsection*{b)}
Da die Erwatungswerte erster und zweiter Ordnung der Zufallsprozesse $X(t)$ und $Y(t)$ invariant gegenüber zeitlicher Verschiebung sind, sind diese schwach stationär.
\end{document} 
