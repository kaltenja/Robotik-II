%%%%%%%%%%%%%%%%%%%%%%%%%%%%%%%%%%%%%%%%%
% Wenneker Assignment
% LaTeX Template
% Version 2.0 (12/1/2019)
%
% This template originates from:
% http://www.LaTeXTemplates.com
%
% Authors:
% Vel (vel@LaTeXTemplates.com)
% Frits Wenneker
%
% License:
% CC BY-NC-SA 3.0 (http://creativecommons.org/licenses/by-nc-sa/3.0/)
% 
%%%%%%%%%%%%%%%%%%%%%%%%%%%%%%%%%%%%%%%%%

%----------------------------------------------------------------------------------------
%	PACKAGES AND OTHER DOCUMENT CONFIGURATIONS
%----------------------------------------------------------------------------------------

\documentclass[11pt]{scrartcl} % Font size

\input{structure.tex} % Include the file specifying the document structure and custom commands

%----------------------------------------------------------------------------------------
%	TITLE SECTION
%----------------------------------------------------------------------------------------

\title{	
	\normalfont\normalsize
	\textsc{Universität Würzburg}\\ % Your university, school and/or department name(s)
	\vspace{25pt} % Whitespace
	\rule{\linewidth}{0.5pt}\\ % Thin top horizontal rule
	\vspace{20pt} % Whitespace
	{\huge Übung 10: Zufallsvariablen und Zufallsprozesse}\\ % The assignment title
	\vspace{12pt} % Whitespace
	\rule{\linewidth}{2pt}\\ % Thick bottom horizontal rule
	\vspace{12pt} % Whitespace
}

\author{\LARGE Alexander Björk, Janis Kaltenthaler} % Your name

\date{\normalsize\today} % Today's date (\today) or a custom date

\begin{document}

\maketitle % Print the title


\section*{Aufgabe 10‐1: Statistische Unabhängigkeit (2 Punkte)} 
Wir berechnen zuerst die Randverteilungsdichtefunktionen $f_X(x)$ und $f_Y(y)$:
\begin{align*}
	f_X(x)&=\int_{-\infty}^{\infty}f_{XY}(x,y)\hspace{3pt}dy=\begin{cases}\int_{0}^{\infty}e^{-(x+y)}\hspace{3pt}dy=e^{-x}, & x\ge\text{ und }y \ge0 \\0, & \text{sonst} \end{cases}\\
	f_Y(y)&=\int_{-\infty}^{\infty}f_{XY}(x,y)\hspace{3pt}dx=\begin{cases}\int_{0}^{\infty}e^{-(x+y)}\hspace{3pt}dx=e^{-y}, & x\ge\text{ und }y \ge0 \\0, & \text{sonst} \end{cases}
\end{align*}
Die Zufallsvariablen $X$ und $Y$ sind genau dann unabhängig, falls $f_X(x)\cdot f_Y(y)$ eine Verbundverteilunsgdichtefunktion von $X$ und $Y$ ist:
\begin{align*}
	f_X(x)\cdot f_Y(y)=\begin{cases}e^{-x}\cdot e^{-y}=e^{-(x+y)}, & x\ge\text{ und }y \ge0 \\0, & \text{sonst} \end{cases}=f_{XY}(x,y)
\end{align*}
Da $f_X(x)\cdot f_Y(y)=f_{XY}(x,y)$ gilt, sind die Zufallsvariablen $X$ und $Y$ statistisch unabhängig voneinander.


\section*{Aufgabe 10‐2: Zufallsprozesse (2,5Punkte)}
\subsection*{a)}
\begin{align*}
\lim_{N \rightarrow \infty} \dfrac{1}{N} \sum_{i=1}^N X_i\left(t\right) = \lim_{T \rightarrow \infty} \dfrac{1}{2T} \int_{-T}^T X_j\left(t\right)dt \quad , \quad \text{für beliebige j}
\end{align*}

Dies lässt sich folgendermaßen interpretieren. Für welche Scharen ist der Erwartungswert erster Ordnung (Scharmittelwert für bestimmten Zeitpunkt) gleich dem Zeitmittelwert einer beliebiger Musterfunktion selbiger Schar.
Dies kann zutreffen für:
\begin{itemize}
\item \textbf{Schar 1} könnte hier zutreffend sein, da sowohl Erwartungswert als auch Zeitmittelwert ungefähr 0 ist.
\item Auch \textbf{Schar 2} könnte passen, da hier die selben Bedingungen für Größen erster Ordnung gelten wie bei Schar 1.
\item Ebenso ist \textbf{Schar 4} möglich, es handelt sich nur um eine Verschiebung um $+1$. 
\end{itemize}
Schar 3 erfüllt diese Eigenschaft womöglich nicht, da weder über den Scharmittelwert noch über den Zeitmittelwert eine Aussage getroffen werden kann. Schar 5 hat über die Musterfunktionen einen sinusförmigen Verlauf. Der Zeitmittelwert ist jedoch 0.

\subsection*{b)}
\begin{align*}
\lim_{N \rightarrow \infty} \dfrac{1}{N} \sum_{i=1}^N X_i^2\left(t\right) = \lim_{T \rightarrow \infty} \dfrac{1}{2T} \int_{-T}^T X_j^2\left(t\right)dt \quad , \quad \text{für beliebige j}
\end{align*}
Dies lässt sich durch "Erwartungswert zweiter Ordnung (quadratischer Scharmittelwert) ist gleich quadratischem Zeitmittelwert einer beliebigen Musterfunktion selbiger Schar" ausdrücken.

Dies kann zutreffen für:
\begin{itemize}
\item \textbf{Schar 4} könnte diese Eigenschaft erfüllen, da der Zufallsprozess scheinbar Zahlen um 1 generiert und $1^2=1$ ist.
\item Je nachdem wie die weiteren Musterfunktionen von \textbf{Schar 3} verlaufen, kann hier die Eigenschaft gelten. Durch die Quadrierung könnte hier Schar- und Zeitmittelwert gleich sein. Eine konkrete Aussage lässt sich aber bei nur 3 Musterfunktionen nicht treffen.
\end{itemize}
Für Schar 1 und 2 lässt sich nichts aussagen, da durch die Quadrierung und der fehlenden Beschriftung der Ordinate ein nicht-vorhersehbares Verhalten entsteht. Unter der Voraussetzung, dass Schar 5 ein sinusförmigen Verlauf hat, lässt sich diese Schar auch hier ausschließen. Für den quadratischen Scharmittelwert kann man $\sin^2(x)$ annehmen und für den quadratischen Zeitmittelwert somit $\dfrac{x - \sin x \cos x}{2}$.


\section*{Aufgabe 10-3: Stationäre und ergodische Zufallsprozesse (2,5Punkte)}
\subsection*{a)}
\begin{itemize}
\item \textbf{Schar 1} ist schwach stationär. Erwartungswerte erster und zweiter Ordnung sind konstant und daher unabhängig von der Zeit.
\item Selbiges gilt für \textbf{Schar 2, 3 und 4}. 
\end{itemize}
Schar 5 hat offensichtlich keinen zeitunabhängigen Erwartungswert erster Ordnung und ist somit nicht schwach stationär.

\subsection*{b)}
\begin{itemize}
\item \textbf{Schar 1 und 2} sind ergodisch. Sie sind schwach stationär und sowohl die Zeitmittelwerte als auch die Scharmittelwerte stimmen überein (ungefähr 0).
\item \textbf{Schar 4} ist ebenfalls ergodisch. Es gilt die gleiche Begründung wie für Schar 1 und 2, nur dass hier die Mittelwerte ungefähr 1 sind.
\end{itemize}
Schar 3 ist nicht ergodisch, weil die Zeitmittelwerte über die Musterfunktionen nicht gleich sind. Schar 5 ist auch nicht ergodisch, weil sie nicht (schwach) stationär ist.


\section*{Aufgabe 10‐4: Erwartungswerte von Zufallsprozessen (1,5 Punkte)}
\subsection*{a)}
Der Erwartungswert $E[Z(t)]=\mu_z$ des Zufallsprozesses $Z(t)=a\cdot X(t)+b\cdot Y(t)$ erhält man wie folgt:
\begin{align*}
	E[Z(t)]=a\cdot E[X(t)]+b\cdot E[Y(t)]=2\cdot E[X(t)]+3\cdot E[Y(t)]=3
\end{align*}
Die Varianzen der Zufallsprozesse erhält man wie folgt:
\begin{align*}
	\text{Var}[X(t)]&=E[X^2(t)]-(E[X(t)])^2=6\\
	\text{Var}[Y(t)]&=E[Y^2(t)]-(E[Y(t)])^2=2
\end{align*}
Die Varianz des Zufallsprozesses $Z(t)$ erhält man wie folgt, wobei durch die Unkorreliertheit $\text{Cov}[X(t),Y(t)]=0$ gilt.
\begin{align*}
\text{Var}[Z(t)]=a^2\cdot\text{Var}[X(t)]+2ab\cdot\text{Cov}[X(t),Y(t)]+b^2\cdot\text{Var}[Y(t)]=42
\end{align*}
Das gewöhnliche Moment zweiter Ordnung erhalten wir mithilfe der Varianz und des Erwarungswertes wie folgt:
\begin{align*}
		E[Z^2(t)]=\text{Var}[Z(t)]+(E[Z(t)])^2=51
\end{align*}
\subsection*{b)}
Da die Erwatungswerte erster und zweiter Ordnung der Zufallsprozesse $X(t)$ und $Y(t)$ invariant gegenüber zeitlicher Verschiebung sind, sind diese schwach stationär.
\end{document} 
