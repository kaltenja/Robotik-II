%%%%%%%%%%%%%%%%%%%%%%%%%%%%%%%%%%%%%%%%%
% Wenneker Assignment
% LaTeX Template
% Version 2.0 (12/1/2019)
%
% This template originates from:
% http://www.LaTeXTemplates.com
%
% Authors:
% Vel (vel@LaTeXTemplates.com)
% Frits Wenneker
%
% License:
% CC BY-NC-SA 3.0 (http://creativecommons.org/licenses/by-nc-sa/3.0/)
% 
%%%%%%%%%%%%%%%%%%%%%%%%%%%%%%%%%%%%%%%%%

%----------------------------------------------------------------------------------------
%	PACKAGES AND OTHER DOCUMENT CONFIGURATIONS
%----------------------------------------------------------------------------------------

\documentclass[11pt]{scrartcl} % Font size

\input{structure.tex} % Include the file specifying the document structure and custom commands

%----------------------------------------------------------------------------------------
%	TITLE SECTION
%----------------------------------------------------------------------------------------

\title{	
	\normalfont\normalsize
	\textsc{Universität Würzburg}\\ % Your university, school and/or department name(s)
	\vspace{25pt} % Whitespace
	\rule{\linewidth}{0.5pt}\\ % Thin top horizontal rule
	\vspace{20pt} % Whitespace
	{\huge Übung 3}\\ % The assignment title
	{\Large Steuerbarkeit, Beobachtbarkeit \& Stabilität}\\
	\vspace{12pt} % Whitespace
	\rule{\linewidth}{2pt}\\ % Thick bottom horizontal rule
	\vspace{12pt} % Whitespace
}

\author{\LARGE Alexander Björk, Janis Kaltenthaler} % Your name

\date{\normalsize\today} % Today's date (\today) or a custom date

\begin{document}

\maketitle % Print the title

%----------------------------------------------------------------------------------------
%	FIGURE EXAMPLE
%----------------------------------------------------------------------------------------

\section*{Aufgabe 3-1. Steuerbarkeit und Stabilität (5 Punkte)}
\subsection*{a)}
Um das System auf Steuerbarkeit zu untersuchen wird nach dem Steuerbarkeitskriterium von Kalman geprüft. Dies besagt, dass ein System steuerbar ist, wenn die Steuerbarkeitsmatrix $S_S$ den Rang $n$ hat.
\begin{align*}
\text{Rang} \hspace{3pt} S_S = n
\end{align*}
Bei Eingrößensystem genügt eine Prüfung von
\begin{align*}
\text{det} \hspace{3pt} S_S \neq 0.
\end{align*}
Die Steuerbarkeitsmatrix ist hier:
\begin{align*}
S_S=\begin{bmatrix}
b & Ab
\end{bmatrix} =
\begin{bmatrix}
8 & -64 \\
24 & -192
\end{bmatrix}
\end{align*}
Die Determinante ist
\begin{align*}
\text{det} \hspace{3pt} S_S=8\cdot (-192) - (-64) \cdot 24 = 0.
\end{align*}
Das System ist somit nicht vollständig steuerbar.

\subsection*{b)}
Für die einfache Berechnung der Übertragungsfunktion in Pol-Nullstellenform, kann man zuvor die Pol- und Nullstellen aus der Polynomform der Übertragungsfunktion berechnen.\\
Die Übertragungsfunktion ist (für die Berechnung aus dem Zustandsraummodell) definiert als
\begin{align*}
G(s) = c^T(sI-A)^{-1}b+d.
\end{align*}
Daraus ergibt sich für unseren Fall:
\begin{align*}
G(s)=
\begin{bmatrix}
1 & 0
\end{bmatrix}
\left(
\begin{bmatrix}
s-4 & 4\\
0 & s+8
\end{bmatrix}
\right)^{-1}
\begin{bmatrix}
8\\
24
\end{bmatrix}
=
\dfrac{8s-32}{s^2+4s-32}
\end{align*}
Dies lässt sich zu
\begin{align} \tag{1}
\label{eqn:Gl1}
G(s)=8\dfrac{s-4}{(s-4)(s+8)}
\end{align}
umformen. Setzt man voraus, dass $s \neq 4$ gilt, lässt sich dies sogar noch weiter kürzen zu
\begin{align} \tag{2}
\label{eqn:Gl2}
G(s) = 8 \dfrac{1}{s+8}.
\end{align}
\subsection*{c)}
\label{sec:1c}
Berechnet man nun die Pol- und Nullstellen nach Gl. \ref{eqn:Gl2}, erkennt man, dass das System nach dieser Übertragungsfunktion keine Nullstellen und nur eine Polstelle bei $s_p=-8$ hat. Die Ordnung ist somit 1.

\subsection*{d)}
Die Berechnung der Pol- und Nullstellen mit Hilfe von z.B. MATLAB zeigt, dass das System anders als in Gl. \ref{eqn:Gl2} gezeigt, eine höhere Ordnung besitzt. Es gibt sehr wohl eine Pol- als auch eine Nullstelle bei $s=4$.\\
Das Problem ist hier das zu weite Kürzen von Gl. \ref{eqn:Gl1} zu Gl. \ref{eqn:Gl2}. Bei diesem Vorgang verliert man die Pol- und Nullstelle bei $s=4$, sowie eine Ordnung.

\subsection*{e)}


\section*{Aufgabe 3-2. Steuerbarkeit und Beobachtbarkeit eines elektrischen Systems (5 Punkte)}
\subsection*{a)}
\subsection*{b)}
\subsection*{c)}
\subsection*{d)}
\subsection*{e)}

\end{document}
