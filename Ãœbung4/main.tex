%%%%%%%%%%%%%%%%%%%%%%%%%%%%%%%%%%%%%%%%%
% Wenneker Assignment
% LaTeX Template
% Version 2.0 (12/1/2019)
%
% This template originates from:
% http://www.LaTeXTemplates.com
%
% Authors:
% Vel (vel@LaTeXTemplates.com)
% Frits Wenneker
%
% License:
% CC BY-NC-SA 3.0 (http://creativecommons.org/licenses/by-nc-sa/3.0/)
% 
%%%%%%%%%%%%%%%%%%%%%%%%%%%%%%%%%%%%%%%%%

%----------------------------------------------------------------------------------------
%	PACKAGES AND OTHER DOCUMENT CONFIGURATIONS
%----------------------------------------------------------------------------------------

\documentclass[11pt]{scrartcl} % Font size

%%%%%%%%%%%%%%%%%%%%%%%%%%%%%%%%%%%%%%%%%
% Wenneker Assignment
% Structure Specification File
% Version 2.0 (12/1/2019)
%
% This template originates from:
% http://www.LaTeXTemplates.com
%
% Authors:
% Vel (vel@LaTeXTemplates.com)
% Frits Wenneker
%
% License:
% CC BY-NC-SA 3.0 (http://creativecommons.org/licenses/by-nc-sa/3.0/)
% 
%%%%%%%%%%%%%%%%%%%%%%%%%%%%%%%%%%%%%%%%%

%----------------------------------------------------------------------------------------
%	PACKAGES AND OTHER DOCUMENT CONFIGURATIONS
%----------------------------------------------------------------------------------------

\usepackage{amsmath, amsfonts, amsthm} % Math packages

\usepackage{listings} % Code listings, with syntax highlighting

\usepackage[english]{babel} % English language hyphenation

\usepackage{graphicx} % Required for inserting images
\usepackage{caption}
\graphicspath{{Figures/}{./}} % Specifies where to look for included images (trailing slash required)

\usepackage{booktabs} % Required for better horizontal rules in tables

\numberwithin{equation}{section} % Number equations within sections (i.e. 1.1, 1.2, 2.1, 2.2 instead of 1, 2, 3, 4)
\numberwithin{figure}{section} % Number figures within sections (i.e. 1.1, 1.2, 2.1, 2.2 instead of 1, 2, 3, 4)
\numberwithin{table}{section} % Number tables within sections (i.e. 1.1, 1.2, 2.1, 2.2 instead of 1, 2, 3, 4)

\setlength\parindent{0pt} % Removes all indentation from paragraphs

\usepackage{enumitem} % Required for list customisation
\setlist{noitemsep} % No spacing between list items

%----------------------------------------------------------------------------------------
%	DOCUMENT MARGINS
%----------------------------------------------------------------------------------------

\usepackage{geometry} % Required for adjusting page dimensions and margins

\geometry{
	paper=a4paper, % Paper size, change to letterpaper for US letter size
	top=2.5cm, % Top margin
	bottom=3cm, % Bottom margin
	left=3cm, % Left margin
	right=3cm, % Right margin
	headheight=0.75cm, % Header height
	footskip=1.5cm, % Space from the bottom margin to the baseline of the footer
	headsep=0.75cm, % Space from the top margin to the baseline of the header
	%showframe, % Uncomment to show how the type block is set on the page
}

%----------------------------------------------------------------------------------------
%	FONTS
%----------------------------------------------------------------------------------------

\usepackage[utf8]{inputenc} % Required for inputting international characters
\usepackage[T1]{fontenc} % Use 8-bit encoding

\usepackage{fourier} % Use the Adobe Utopia font for the document

%\usepackage[framed,numbered,autolinebreaks,useliterate]{mcode}

%----------------------------------------------------------------------------------------
%	SECTION TITLES
%----------------------------------------------------------------------------------------

\usepackage{sectsty} % Allows customising section commands

\sectionfont{\normalfont\bfseries} % \section{} styling
\subsectionfont{\normalfont\bfseries} % \subsection{} styling
\subsubsectionfont{\normalfont\itshape} % \subsubsection{} styling
\paragraphfont{\normalfont\scshape} % \paragraph{} styling

%----------------------------------------------------------------------------------------
%	HEADERS AND FOOTERS
%----------------------------------------------------------------------------------------

\usepackage{scrlayer-scrpage} % Required for customising headers and footers

\ohead*{} % Right header
\ihead*{} % Left header
\chead*{} % Centre header

\ofoot*{} % Right footer
\ifoot*{} % Left footer
\cfoot*{\pagemark} % Centre footer

% MY PACKAGES
%\usepackage[framed,numbered,autolinebreaks,useliterate]{mcode}
\usepackage{listings}
\usepackage{float}
\usepackage{amsmath}
\usepackage{tikz}
\usetikzlibrary{shapes,arrows,positioning}
\usepackage{hyperref} % Include the file specifying the document structure and custom commands

%----------------------------------------------------------------------------------------
%	TITLE SECTION
%----------------------------------------------------------------------------------------

\title{	
	\normalfont\normalsize
	\textsc{Universität Würzburg}\\ % Your university, school and/or department name(s)
	\vspace{25pt} % Whitespace
	\rule{\linewidth}{0.5pt}\\ % Thin top horizontal rule
	\vspace{20pt} % Whitespace
	{\huge Übung 4}\\ % The assignment title
	{\Large Steuerbarkeitund Beobachtbarkeit}\\
	\vspace{12pt} % Whitespace
	\rule{\linewidth}{2pt}\\ % Thick bottom horizontal rule
	\vspace{12pt} % Whitespace
}

\author{\LARGE Alexander Björk, Janis Kaltenthaler} % Your name

\date{\normalsize\today} % Today's date (\today) or a custom date

\begin{document}

\maketitle % Print the title

\tikzstyle{block} = [draw, fill=blue!20, rectangle, 
    minimum height=3em, minimum width=6em]
\tikzstyle{sum} = [draw, fill=blue!20, circle, node distance=1cm]
\tikzstyle{input} = [coordinate]
\tikzstyle{output} = [coordinate]
\tikzstyle{pinstyle} = [pin edge={to-,thin,black}]
\newcommand{\inte}{$\displaystyle \int$}

\section*{Aufgabe4-1. Steuerbarkeit und Beobachtbarkeit von Reihen-und Parallelschaltung (6Punkte)}
Zuerst betrachten wir die \textbf{Reihenschaltung} zweier Integretoren.

\begin{figure}[H]
	\centering
	\begin{tikzpicture}[auto, node distance=2cm,>=latex']
   		% We start by placing the blocks
    	\node [input, name=input] {};
    	\node [block, right of=input] (integrator1) {\inte};
    	\node [block, right = 1.5cm of integrator1] (integrator2) {\inte};
    	\node [output, right of=integrator2] (output) {};

    	% Once the nodes are placed, connecting them is easy. 
    	\draw [draw,->] (input) -- node {$U$} (integrator1);
    	\draw [->] (integrator1) -- node {} (integrator2);
    	\draw [->] (integrator2) -- node [name=y] {$Y$}(output);
	\end{tikzpicture}
	\captionsetup{labelformat=empty}
	\caption{Abb.4-1.1: Blockschaltbild der Reihenschaltung zweier Integratoren.}
\end{figure}
Nach
\begin{align*}
G(s)=\left( c^T \left( sI-A \right)^{-1}b+d \right)
\end{align*}
ist die Übertragungsfunktion eines Integrators
\begin{align*}
G_{\text{int}}(s)=\dfrac{1}{T_is}=\dfrac{K}{s} \quad\quad \text{mit} \quad\quad K=\dfrac{1}{T_i}.
\end{align*}
Demnach ist die Übertragungfunktion einer Reihenschaltung zweier Integratoren
\begin{align*}
G(s) = G_{\text{int}}(s) \cdot G_{\text{int}}(s) = \dfrac{K^2}{s^2}.
\end{align*}
Daraus kann nun das Zustandsraummodell abgelesen werden\footnote{\url{https://lpsa.swarthmore.edu/Representations/SysRepTransformations/TF2SS.html}}, um dann auf Steuerbarkeit und Beobachtbarkeit zu prüfen.\\
Das Zustandsraummodell ist
\begin{align*}
\dot{x} &= \begin{bmatrix}
0 & 1\\
0 & 0
\end{bmatrix} x(t) + \begin{bmatrix}
0 \\
1
\end{bmatrix} u(t) \\
y(t) &= \begin{bmatrix}
K^2 & 0
\end{bmatrix} x(t)
\end{align*}
Um jetzt auf Steuerbarkeit und Beobachtbarkeit zu prüfen, müssen nun die Steuerbarkeits- und Beobachtbarkeitsmatrizen gebildet werden:
\begin{align*}
S_S &= \begin{bmatrix}
B & AB & A^2B & ... & A^{n-1}B
\end{bmatrix} = \begin{bmatrix}
B & AB
\end{bmatrix} = \begin{bmatrix}
0 & 1\\
1 & 0
\end{bmatrix} \\
S_B &= \begin{bmatrix}
C & CA & CA^2 & ... & CA^{n-1}
\end{bmatrix}^T = \begin{bmatrix}
C \\
CA
\end{bmatrix} = \begin{bmatrix}
K^2 & 0\\
0 & K^2
\end{bmatrix}
\end{align*}

Untersucht man nun deren Determinanten, können Rückschlüsse auf die Steuerbarkeit und Beobachtbarkeit gezogen werden. Die Determinante der Steuerbarkeitsmatrix ist
\begin{align*}
\text{det}\left( S_S \right) &= 0 - 1 = -1 \neq 0.
\end{align*}
Die Determinante der Beobachtbarkeitsmatrix ist
\begin{align*}
\text{det}\left( S_B \right) &= K^2 \cdot K^2 - 0 = K^4 \neq 0.
\end{align*}
Beide sind ungleich 0. Das System ist damit vollständig steuerbar und beobachtbar. \\

Nun betrachten wir die \textbf{Parallelschaltung} zweier Integratoren.
\begin{figure}[H]
	\centering
	\begin{tikzpicture}[auto, node distance=2cm,>=latex']
   		% We start by placing the blocks
    	\node [input, name=input] {};
    	\node [input, name=input2, right of = input] {};
    	\node [block, above right = of input2] (integrator1) {\inte};
    	\node [block, below right = of input2] (integrator2) {\inte};
    	\node [sum, right = 5cm of input2] (sum) {};
    	\node [output, right of=sum] (output) {};

    	% Once the nodes are placed, connecting them is easy. 
    	\draw [draw, pos = 0.75] (input) |- node {$U$} (input2);
    	\draw [draw,->] (input2) |- node {} (integrator1);
    	\draw [draw,->] (input2) |- node {} (integrator2);
    	\draw [draw,->] (integrator1) -| node {} (sum);
    	\draw [draw,->] (integrator2) -| node {} (sum);
    	\draw [->] (sum) -- node [name=y] {$Y$}(output);
	\end{tikzpicture}
	\captionsetup{labelformat=empty}
	\caption{Abb.4-1.2: Blockschaltbild der Parallelschaltung zweier Integratoren.}
\end{figure}

Die Übertragungsfunktion ist
\begin{align*}
G(s) = G_{\text{int}}(s) + G_{\text{int}}(s) = 2\dfrac{K}{s}.
\end{align*}
Daraus kann erneut das Zustandsraummodell abgelesen werden.
\begin{align*}
\dot{x} &= u(t) \\
y(t) &= 2K x(t)
\end{align*}

Nun ist es wieder möglich, die Steuerbarkeits- und die Beobachtbarkeitsmatrix zu bilden.
\begin{align*}
S_S &= \left[B\right]=1 \neq 0 \\
S_B &= \left[C\right] = 2K \neq 0
\end{align*}
Damit ist auch die Parallelschaltung zweier Integratoren vollständig steuerbar und beobachtbar.

\section*{Aufgabe4-2. Steuerbarkeit und Beobachtbarkeit eines Mehrgrößensystems (3Punkte)}
Die Prüfung auf Steuerbarkeit und Beobachtbarkeit bei Mehrgrößensystemen ist sehr ähnlich zu der von Eingrößensystemen. Man bildet zuerst die benötigten Matrizen:
\begin{align*}
S_S &= \begin{bmatrix}
B & AB
\end{bmatrix} = \begin{bmatrix}
2 & 1 & -6 & -3\\
1 & 0 & -2 & 0
\end{bmatrix} \\
S_B &= \begin{bmatrix}
C & CA
\end{bmatrix}^T = \begin{bmatrix}
2 & 0\\
1 & 0\\
-6 & 0\\
-3 & 0
\end{bmatrix}
\end{align*}

Und dann die zugehörigen Transponierten:

\begin{align*}
{S_S}^T &= \begin{bmatrix}
2 & 1\\
1 & 0\\
-6 & -2\\
-3 & 0
\end{bmatrix} \\
{S_B}^T &= \begin{bmatrix}
2 & 1 & -6 & -3\\
0 & 0 & 0 & 0
\end{bmatrix}
\end{align*}

Multipliziert man diese und bildet dann die Determinante, kann man Rückschlüsse auf die Steuerbarkeit und Beobachtbarkeit des Systems ziehen.

\begin{align*}
S_S{S_S}^T&=\begin{bmatrix}
50 & 14\\
14 & 5
\end{bmatrix}\\
\text{det}\left( S_S{S_S}^T \right) &= 54 \neq 0\\
{S_B}^TS_B&=\begin{bmatrix}
50 & 0\\
0 & 0
\end{bmatrix}\\
\text{det}\left( {S_B}^TS_B \right) &= 0
\end{align*}

Das System ist somit vollständig steuerbar, jedoch nicht vollständig beobachtbar.


\section*{Aufgabe4-3. Steuerbarkeitund Beobachtbarkeitdes Pendels am Wagen (4Punkte)}
\subsection*{a)}

\subsection*{b)}

\subsection*{c)}

\subsection*{d)}

\end{document}
