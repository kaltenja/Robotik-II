%%%%%%%%%%%%%%%%%%%%%%%%%%%%%%%%%%%%%%%%%
% Wenneker Assignment
% LaTeX Template
% Version 2.0 (12/1/2019)
%
% This template originates from:
% http://www.LaTeXTemplates.com
%
% Authors:
% Vel (vel@LaTeXTemplates.com)
% Frits Wenneker
%
% License:
% CC BY-NC-SA 3.0 (http://creativecommons.org/licenses/by-nc-sa/3.0/)
% 
%%%%%%%%%%%%%%%%%%%%%%%%%%%%%%%%%%%%%%%%%

%----------------------------------------------------------------------------------------
%	PACKAGES AND OTHER DOCUMENT CONFIGURATIONS
%----------------------------------------------------------------------------------------

\documentclass[11pt]{scrartcl} % Font size

\input{structure.tex} % Include the file specifying the document structure and custom commands

%----------------------------------------------------------------------------------------
%	TITLE SECTION
%----------------------------------------------------------------------------------------

\title{	
	\normalfont\normalsize
	\textsc{Universität Würzburg}\\ % Your university, school and/or department name(s)
	\vspace{25pt} % Whitespace
	\rule{\linewidth}{0.5pt}\\ % Thin top horizontal rule
	\vspace{20pt} % Whitespace
	{\huge Übung 8}\\ % The assignment title
	{\Large Beobachter}\\
	\vspace{12pt} % Whitespace
	\rule{\linewidth}{2pt}\\ % Thick bottom horizontal rule
	\vspace{12pt} % Whitespace
}

\author{\LARGE Alexander Björk, Janis Kaltenthaler} % Your name

\date{\normalsize\today} % Today's date (\today) or a custom date

\begin{document}

\maketitle % Print the title

\tikzstyle{block} = [draw, fill=blue!20, rectangle, 
    minimum height=3em, minimum width=6em]
\tikzstyle{sum} = [draw, fill=blue!20, circle, node distance=1cm]
\tikzstyle{input} = [coordinate]
\tikzstyle{output} = [coordinate]
\tikzstyle{pinstyle} = [pin edge={to-,thin,black}]
\newcommand{\inte}{$\displaystyle \int$}

\section*{Aufgabe 8-1. Beobachterentwurf (3 Punkte)}
\subsection*{a)}
Zu prüfen ist ob die Regelstrecke vollständig steuerbar und vollständig beobachtbar ist:
\subsubsection*{Steuerbarkeit}
\begin{align*}
	S_S&=\begin{bmatrix}b&Ab \end{bmatrix}=\begin{bmatrix}0&-0.5\\0.5&-1 \end{bmatrix}\\
	\text{det}\hspace{3pt}S_S&=0.25\neq0
\end{align*}
Die Regelstrecke ist vollständig Steuerbar.
\subsubsection*{Beobachtbarkeit}
\begin{align*}
	S_B&=\begin{bmatrix}C\\CA \end{bmatrix}=\begin{bmatrix}0&1\\1&-2 \end{bmatrix}\\
	\text{det}\hspace{3pt}S_B&=-1\neq0
\end{align*}
Die Regelstrecke ist vollsträndig beobachtbar.
\subsection*{b)}
Aufgrund der Strucktur des gegebenen Zustandsraummodells wissen wir dass dieses in der Beobachtungsnormalform vorliegt.\\
Die geforderten Eigenwerte des Beobachtermodells sind mit $\lambda_{B1}=-4$ und $\lambda_{B2}=-3$ gegeben. Daraus ergibt sich folgendes charakteristisches Polynom:
\begin{align*}
	p(\lambda_B)=\lambda^2+\underbrace{7}_{a_{B1}}\cdot\lambda+\underbrace{12}_{a_{B_0}}
\end{align*}
Die Koeffizienten $a_0=1$ und $a_1=2$ des charakteristischen Polynomes der Regelstrecke erhalten wird durch Ablesen aus dem gegebenen Zustandsraummodells.\\
Wie folgt kann nun die Beobachterrückführung berechnet werden:
\begin{align*}
	l^T=\begin{bmatrix}a_{B0}&a_{B1}\end{bmatrix}-\begin{bmatrix}a_{0}&a_{1}\end{bmatrix}=\begin{bmatrix}11&5\end{bmatrix}
\end{align*}

%Der Luenbergerbeobachter für das gegebene System hat nun die folgende Form:
%\begin{align*}
%	\dot{\hat{x}}(t)&=(A-lc^t)\hat{x}(t)+bu(t)+ly(t)\\
%	\dot{\hat{x}}(t)&=\begin{bmatrix}0&-12\\1&-7\end{bmatrix}\hat{x}(t)+\begin{bmatrix}0\\0.5\end{bmatrix}u(t)+\begin{bmatrix}11\\5\end{bmatrix}y(t)
%\end{align*}
Das Zustandsraummodell der Regelstrecke und des Beobachters erhält man wie folgt:
\begin{align*}
	\dot{\tilde{x}}(t)&=\begin{bmatrix}\dot{x}(t)\\\dot{\hat{x}}(t)\end{bmatrix}=\begin{bmatrix}A&0\\lc^T&A-lc^T\end{bmatrix}\cdot\begin{bmatrix}x(t)\\\hat{x}(t)\end{bmatrix}+\begin{bmatrix}B&E\\B&0\end{bmatrix}\cdot \begin{bmatrix}u(t)\\d(t)\end{bmatrix}\\
	y(t)&=\begin{bmatrix}C&0\end{bmatrix}\cdot\begin{bmatrix}x(t)\\\hat{x}(t)\end{bmatrix}\\\\
	\dot{\tilde{x}}(t)&=\begin{bmatrix}\dot{x}(t)\\\dot{\hat{x}}(t)\end{bmatrix}=\begin{bmatrix}0&-1&0&0\\1&-2&0&0\\0&11&0&-12\\0&5&1&-7\end{bmatrix}\cdot\begin{bmatrix}x(t)\\\hat{x}(t)\end{bmatrix}+\begin{bmatrix}0&0\\0.5&1\\0&0\\0.5&0\end{bmatrix}\cdot \begin{bmatrix}u(t)\\d(t)\end{bmatrix}\\
	y(t)&=\begin{bmatrix}0&1&0&0\end{bmatrix}\cdot\begin{bmatrix}x(t)\\\hat{x}(t)\end{bmatrix}
\end{align*}
\subsection*{c)}
\begin{figure}[H]
\centering
\includegraphics[width=\textwidth]{ohneStoergroesse.png}
\captionsetup{labelformat=empty}
\caption{Abb. 8-1.1: Sprungantwort ohne Störgröße.}
\end{figure}
Etwa zum Zeitpunkt $t=2s$ stimmt der geschätzte Zustand mit dem tatsächlichen Zustand überein.

\subsection*{d)}
\begin{figure}[H]
\centering
\includegraphics[width=\textwidth]{sprungfoermigeStoergroesse.png}
\captionsetup{labelformat=empty}
\caption{Abb. 8-1.2: Sprungantwort mit sprungförmiger Störgröße.}
\end{figure}

\begin{figure}[H]
\centering
\includegraphics[width=\textwidth]{impulsfoermigeStoergroesse.png}
\captionsetup{labelformat=empty}
\caption{Abb. 8-1.3: Sprungantwort mit impulsförmiger Störgröße.}
\end{figure}

\begin{figure}[H]
\centering
\includegraphics[width=\textwidth]{normalverteilteStoergroesse.png}
\captionsetup{labelformat=empty}
\caption{Abb. 8-1.4: Sprungantwort mit normalverteilter Störgröße.}
\end{figure}

Wirkt eine sprungförmige Störgröße auf die Regelstrecke, so bleibt eine stetige Abweichung zwischen dem geschätzten Zustand der ersten Zustandsvariable und dem tatsächlichen Zustand.\\
Wirkt eine normalverteilte oder impulsförmige Störgröße auf die Regelstrecke, so nähert sich sich der geschätzte Zustand dem tatsächlichen Zustand an.


\end{document}
